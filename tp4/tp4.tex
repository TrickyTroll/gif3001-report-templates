\documentclass[]{article}
\usepackage[french]{babel}
\usepackage[autolanguage]{numprint}
\usepackage[T1]{fontenc}
\usepackage{hyphenat}
\usepackage{listings}
\usepackage{booktabs}
\usepackage{lmodern}
\usepackage{amsmath}
\usepackage{xcolor}
\usepackage{hyperref,graphicx, caption} 
\lstset{
  basicstyle=\ttfamily,
  columns=fullflexible,
  frame=single,
  breaklines=true,
  postbreak=\mbox{\textcolor{red}{$\hookrightarrow$}\space},
}

\title{Projet 4 -- Analyse de flux TCP}
\author{Équipe 0: Étienne et Florent Parent} % TODO: mettre votre nom
\date{date de remise} % TODO: mettre la date de remise

\begin{document}

\maketitle

\section*{Analyse de flux TCP dans Wireshark}

\subsection*{Établissement de la connexion TCP}

\begin{enumerate}
      \item L'adresse IP du client est \texttt{X.X.X.X} et le numéro de port du client est le
            \texttt{XXXX}. L'adresse IP du serveur est \texttt{X.X.X.X} et le numéro de port du
            serveur est le \texttt{XXXX}.
      \item Le numéro de séquence du paquet SYN utilisé par le client pour établir la connexion
            est le \texttt{XXXX}. Nous avons pu identifier que c'était un paquet SYN parce que...
      \item Le numéro de séquence du paquet SYN-ACK utilisé par le serveur pour confirmer l'établissement
            de la connexion est le \texttt{XXXX}. Nous avons pu identifier que c'était un paquet SYN parce que...
      \item Le numéro d'accusé de réception dans le paquet SYN-ACK utilisé par le serveur pour confirmer
            l'établissement de la connexion est le \texttt{XXXX}. La valeur du champ ACK représente...

      \item Il y a / il n'y a pas d'options TCP échangées lors de l'établissement de la connexion. S'il
            y en a, les voici:
            \begin{itemize}
                  \item Option 1: cette option est...
                  \item Option 2: cette option sert à...
                  \item Option 3: foobar
            \end{itemize}
      
      \item La valeur du MSS du client est \texttt{XXXX} et la valeur du MSS du serveur est \texttt{XXXX}.
            Le client détermine le MSS en... Le MTU est ainsi de \texttt{XXXX} octets.
      \item Au paquet \#3, la taille de la fenêtre de réception annoncée par le client est de \texttt{XXXX}
            octets. Une option \textit{Window Scale} est utilisée / n'est pas utilisée. Sa valeur est de...
      \item Au paquet \#5, la taille de la fenêtre de réception annoncée par le serveur est de \texttt{XXXX}
            octets. Une option \textit{Window Scale} est utilisée / n'est pas utilisée. Sa valeur est de...
            La taille de la fenêtre annoncée par le serveur est / n'est pas la même que celle annoncée par
            le client.
\end{enumerate}

\subsection*{Taux de transmission}

\begin{enumerate}
      \item La taille des données transportées dans chacun des segments TCP est de texttt{XXXX} octets.
            La taille correspond / ne correspond pas à la taille du MSS parce que...

      \item \texttt{XX} segments sont envoyés sans attendre d'accusé de réception. Cela représente \texttt{XX}
            octets.
      \item Le numéro du segment ACK est le \texttt{XXXX}. Le temps d'aller-retour est ainsi de ...
      \item Le taux de transmission de la première séquence est de \texttt{XXXX} Mbit/s.
      \item Le taux de transmission moyen de toute la capture est de \texttt{XXXX} Mbit/s. Voici les détails
            de nos calculs:
            \begin{itemize}
                  \item Temps total de la capture: \texttt{XXXX} secondes
                  \item $x = 3$
                  \item $y = x/4$
            \end{itemize}
      \item Comparaisons des taux de transmission moyens... Au début de la transmission, le client est limité
            par... parce que...
\end{enumerate}

\section*{Contrôle de congestion et contrôle de flux}

\begin{enumerate}
      \item Voici notre analyse détaillée du graphique...
      \item Voici un changement de configuration qui pourrait améliorer le taux de transmission ainsi que nos
            justifications.
\end{enumerate}

\end{document}
