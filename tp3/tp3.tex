\documentclass[]{article}
\usepackage[french]{babel}
\usepackage[autolanguage]{numprint}
\usepackage[T1]{fontenc}
\usepackage{hyphenat}
\usepackage{listings}
\usepackage{booktabs}
\usepackage{lmodern}
\usepackage{amsmath}
\usepackage{xcolor}
\usepackage{hyperref,graphicx, caption} 
\lstset{
  basicstyle=\ttfamily,
  columns=fullflexible,
  frame=single,
  breaklines=true,
  postbreak=\mbox{\textcolor{red}{$\hookrightarrow$}\space},
}

\title{Projet 3 -- Adressage et routage IP}
\author{Équipe 0: Étienne et Florent Parent} % TODO: mettre votre nom
\date{date de remise} % TODO: mettre la date de remise

\begin{document}

\maketitle

\section*{Adressage et configuration d'un réseau IP}

Notre plan d'adressage est présenté dans le tableau~\ref{tab:plan-adressage}.

% À noter que ce tableau peut servir pour présenter le plan d'adressage du projet en entier.
% Il n'est donc pas nécessaire d'en faire un pour chaque section (IPv4, IPv6 et routage OSPF)

\begin{center}
      \begin{table}
            \begin{tabular}{ c c c c c }
                  \toprule
                  \textbf{Appareil} & \textbf{Interface} & \textbf{Adresse IPv4} & \textbf{Masque IPv4} & \textbf{Adresse IPv6}\\
                  \toprule
                  PC1 & \texttt{eth0} & \texttt{A.B.C.D} & \texttt{E.F.G.H} & \texttt{I:J:K:L:M:N:O:P}\\
                  \midrule
                  PC2 & \texttt{eth0} & \texttt{A.B.C.D} & \texttt{E.F.G.H} & \texttt{I:J:K:L:M:N:O:P}\\
                  \midrule
                  R1 & \texttt{eth0} & \texttt{A.B.C.D} & \texttt{E.F.G.H} & \texttt{I:J:K:L:M:N:O:P}\\
                  R1 & \texttt{eth1} & \texttt{A.B.C.D} & \texttt{E.F.G.H} & \texttt{I:J:K:L:M:N:O:P}\\
                  R1 & \texttt{eth2} & \texttt{A.B.C.D} & \texttt{E.F.G.H} & \texttt{I:J:K:L:M:N:O:P}\\
                  \midrule
                  R2 & \texttt{eth0} & \texttt{A.B.C.D} & \texttt{E.F.G.H} & \texttt{I:J:K:L:M:N:O:P}\\
                  R2 & \texttt{eth1} & \texttt{A.B.C.D} & \texttt{E.F.G.H} & \texttt{I:J:K:L:M:N:O:P}\\
                  R2 & \texttt{eth2} & \texttt{A.B.C.D} & \texttt{E.F.G.H} & \texttt{I:J:K:L:M:N:O:P}\\
                  \midrule
                  R3 & \texttt{eth0} & \texttt{A.B.C.D} & \texttt{E.F.G.H} & \texttt{I:J:K:L:M:N:O:P}\\
                  R3 & \texttt{eth1} & \texttt{A.B.C.D} & \texttt{E.F.G.H} & \texttt{I:J:K:L:M:N:O:P}\\
                  R3 & \texttt{eth2} & \texttt{A.B.C.D} & \texttt{E.F.G.H} & \texttt{I:J:K:L:M:N:O:P}\\
            \end{tabular}
            \caption{Plan d'adressage IPv4 et IPv6.}
            \label{tab:plan-adressage}
      \end{table}
\end{center}

\subsection*{Plan d'adressage IPv4 et IPv6}

\begin{enumerate}
      \item La configuration de mon routeur R1 est présentée à la figure~\ref{fig:ipv4-config-r1}. La configuration
            de mon routeur R2 est présentée à la figure~\ref{fig:ipv4-config-r2}.
      \item Les routes statiques cofigurées sur PC1 et PC2 sont présentées à la figure~\ref{fig:routes-pc1-pc2}.
      \item  Le résultat du \texttt{ping} entre PC1 et PC2 est présenté à la figure~\ref{fig:ping-pc1-pc2}.
\end{enumerate}


\begin{figure} 
      \centering
      \begin{lstlisting}
$ ping google.ca
PING google.ca (172.217.13.99) 56(84) bytes of data.
64 bytes from yul02s04-in-f3.1e100.net (172.217.13.99): icmp_seq=1 ttl=119 time=22.0 ms
64 bytes from yul02s04-in-f3.1e100.net (172.217.13.99): icmp_seq=2 ttl=119 time=38.4 ms
64 bytes from yul02s04-in-f3.1e100.net (172.217.13.99): icmp_seq=3 ttl=119 time=197 ms
64 bytes from yul02s04-in-f3.1e100.net (172.217.13.99): icmp_seq=4 ttl=119 time=37.6 ms
64 bytes from yul02s04-in-f3.1e100.net (172.217.13.99): icmp_seq=5 ttl=119 time=43.5 ms
64 bytes from yul02s04-in-f3.1e100.net (172.217.13.99): icmp_seq=6 ttl=119 time=42.4 ms
      \end{lstlisting}
      \caption[]{Ping du PC1 au PC2}
      \label{fig:ping-pc1-pc2}
\end{figure}

\begin{figure} 
      \centering
      \begin{lstlisting}
$ show running-config
...
      \end{lstlisting}
      \caption[]{Configuration du routeur R1 pour la configuration de réseau IPv4.}
      \label{fig:ipv4-config-r1}
\end{figure}

\begin{figure} 
      \centering
      \begin{lstlisting}
$ show running-config 
...
      \end{lstlisting}
      \caption[]{Configuration du routeur R2 pour rla configuration de réseau IPv4.}
      \label{fig:ipv4-config-r2}
\end{figure}

\begin{figure} 
      \centering
      \begin{lstlisting}
$ ip route
...

$ ip route
...
      \end{lstlisting}
      \caption[]{Routes par défaut configurées sur PC1 et PC2.}
      \label{fig:routes-pc1-pc2}
\end{figure}

\subsection*{Configuration du réseau IPv6}

Mon plan d'adressage IPv6 est présenté dans le tableau~\ref{tab:plan-adressage}.

\begin{enumerate}
      \item La configuration de mon routeur R1 est présentée à la figure~\ref{fig:ipv6-config-r1}. La configuration
            de mon routeur R2 est présentée à la figure~\ref{fig:ipv6-config-r2}.
      \item Les routes statiques cofigurées sur PC1 et PC2 sont présentées à la figure~\ref{fig:ipv6-routes-pc1-pc2}.
      \item  Le résultat du \texttt{ping} entre PC1 et PC2 est présenté à la figure~\ref{fig:ping6-pc1-pc2}.
\end{enumerate}

\begin{figure} 
      \centering
      \begin{lstlisting}
$ ping6 google.com
      \end{lstlisting}
      \caption[]{Ping du PC1 au PC2}
      \label{fig:ping6-pc1-pc2}
\end{figure}

\begin{figure} 
      \centering
      \begin{lstlisting}
$ show running-config
...
      \end{lstlisting}
      \caption[]{Configuration du routeur R1 pour la configuration de réseau IPv6.}
      \label{fig:ipv6-config-r1}
\end{figure}

\begin{figure} 
      \centering
      \begin{lstlisting}
$ show running-config 
...
      \end{lstlisting}
      \caption[]{Configuration du routeur R2 pour rla configuration de réseau IPv6.}
      \label{fig:ipv6-config-r2}
\end{figure}

\begin{figure} 
      \centering
      \begin{lstlisting}
$ ip route
...

$ ip route
...
      \end{lstlisting}
      \caption[]{Routes IPv6 autoconfigurées sur PC1 et PC2.}
      \label{fig:ipv6-routes-pc1-pc2}
\end{figure}

\section*{Routage OSPF}

Le tableau~\ref{tab:plan-adressage} a été augmenté pour contenie les adresses des interfaces
du routeur R3.

Avant de pousuivre, nous nous sommes assurés que les routes par défaut on été retirées.

\subsection*{OSPFv2}

\begin{itemize}
      \item Mon réseau fonctionne comme en témoigne le \texttt{ping} entre PC1 et PC2
            présenté à la figure~\ref{fig:ping-pc1-pc2-ospfv2}.
      \item La table de routage du routeur R1 est présentée à la
            figure~\ref{fig:ospfv2-routes-r1}.
      \item La route utilisée pour atteindre le PC2 est...
      \item C'est / ce n'est pas la route la plus courte.
\end{itemize}

\begin{figure} 
      \centering
      \begin{lstlisting}
$ ping google.ca
      \end{lstlisting}
      \caption[]{Ping du PC1 au PC2 pour la section OSPFv2}
      \label{fig:ping-pc1-pc2-ospfv2}
\end{figure}

\begin{figure} 
      \centering
      \begin{lstlisting}
$ show ip route
...
      \end{lstlisting}
      \caption[]{Table de routage du routeur R1 avec OSPFv2.}
      \label{fig:ospfv2-routes-r1}
\end{figure}


\subsection*{OSPFv3}

\begin{itemize}
      \item  Mon réseau fonctionne comme en témoigne le \texttt{ping6} entre PC1 et PC2
             présenté à la figure~\ref{fig:ping6-pc1-pc2-ospfv2}.
      \item  La table de routage du routeur R1 est présentée à la figure~\ref{fig:ospfv3-routes-r1}.
      \item  La route utilisée pour atteindre le PC2 est...
      \item  C'est / ce n'est pas la route la plus courte.
\end{itemize}


\begin{figure} 
      \centering
      \begin{lstlisting}
$ ping6 google.ca
      \end{lstlisting}
      \caption[]{Ping6 du PC1 au PC2 pour la section OSPFv3}
      \label{fig:ping6-pc1-pc2-ospfv2}
\end{figure}


\begin{figure} 
      \centering
      \begin{lstlisting}
$ show ip route
...
      \end{lstlisting}
      \caption[]{Table de routage du routeur R1 avec OSPFv3.}
      \label{fig:ospfv3-routes-r1}
\end{figure}


\subsection*{Routage et changement de lien}

\begin{enumerate}
      \item  Suite à l'ajout du délais, la route utilisée pour atteindre le PC2 est...
      \item Le temps d'aller-retour entre PC1 et PC2 est initialement...
      \item À partir de la table de routage de R2, nous observons que le cout du lien pour rejoindre 
            le réseau du PC1 est...
      \item Il faudrait un cout de ... pour que le lien passant par R3 soit utilisé.
      \item Suite à la modification du cout, le délais a / n'a pas diminué. On peut expliquer cela par...
      \item Les paquets passent / ne passent maintenant (toujours pas) par R3 comme en témoigne la sortie
            de la commande \texttt{traceroute} présentée à la figure~\ref{fig:traceroute-pc1-pc2}.
\end{enumerate}

\begin{figure}
      \centering
      \begin{lstlisting}
$ traceroute google.ca
...
      \end{lstlisting}
      \caption[]{Traceroute du PC1 au PC2 suite à la modification du cout du lien entre R1 et R2.}
      \label{fig:traceroute-pc1-pc2}
\end{figure}

Nous avons maintenant désactivé le lien entre R1 et R3.

\begin{enumerate}
      \item La transmission des paquets a été / n'a pas été interrompue lors du changement de lien.
      \item Les paquets se sont mis / ne se sont pas mis à passer par un chemin alternatif suite à la
            coupure du lien. La séquence de réponses ICMP qui en témoigne est présentée à la figure~\ref{fig:ping-pc1-pc2-coupure}.
      \item Le chemin maintenant emprunté par les paquets entre PC1 et PC2 est... comme en témoigne la sortie
            de la commande \texttt{traceroute} présentée à la figure~\ref{fig:traceroute-pc1-pc2-coupure}.
\end{enumerate}

\begin{figure}
      \centering
      \begin{lstlisting}
$ ping google.ca
...
      \end{lstlisting}
      \caption[]{Ping du PC1 au PC2 avant et suite à la coupure du lien entre R1 et R3.}
      \label{fig:ping-pc1-pc2-coupure}
\end{figure}

\begin{figure}
      \centering
      \begin{lstlisting}
$ traceroute google.ca
...
      \end{lstlisting}
      \caption[]{Traceroute du PC1 au PC2 suite à la coupure du lien entre R1 et R3.}
      \label{fig:traceroute-pc1-pc2-coupure}
\end{figure}

    
\end{document}