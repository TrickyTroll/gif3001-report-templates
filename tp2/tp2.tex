\documentclass[]{article}
\usepackage[french]{babel}
\usepackage[autolanguage]{numprint}
\usepackage{hyphenat}
\usepackage{listings}
\usepackage{lmodern}
\usepackage{amsmath}
\usepackage{xcolor}
\usepackage{hyperref,graphicx} 
\lstset{
  basicstyle=\ttfamily,
  columns=fullflexible,
  frame=single,
  breaklines=true,
  postbreak=\mbox{\textcolor{red}{$\hookrightarrow$}\space},
}

\title{Projet 2 -- Commutateur, Routeur, ARP et DHCP}
\author{Équipe 0: Étienne et Florent Parent} % TODO: mettre votre nom
\date{date de remise} % TODO: mettre la date de remise


\begin{document}

\maketitle

\section*{Configuration}

\begin{enumerate}

    \item Votre réponse à la question 1. Voir résultat du \texttt{ping}~\ref{fig:ping-pc1-pc2}.

    \begin{figure} 
        \centering
        \begin{lstlisting}
$ ping google.ca
PING google.ca (172.217.13.99) 56(84) bytes of data.
64 bytes from yul02s04-in-f3.1e100.net (172.217.13.99): icmp_seq=1 ttl=119 time=22.0 ms
64 bytes from yul02s04-in-f3.1e100.net (172.217.13.99): icmp_seq=2 ttl=119 time=38.4 ms
64 bytes from yul02s04-in-f3.1e100.net (172.217.13.99): icmp_seq=3 ttl=119 time=197 ms
64 bytes from yul02s04-in-f3.1e100.net (172.217.13.99): icmp_seq=4 ttl=119 time=37.6 ms
64 bytes from yul02s04-in-f3.1e100.net (172.217.13.99): icmp_seq=5 ttl=119 time=43.5 ms
64 bytes from yul02s04-in-f3.1e100.net (172.217.13.99): icmp_seq=6 ttl=119 time=42.4 ms
        \end{lstlisting}
        \caption[]{Ping du PC1 au PC2}
        \label{fig:ping-pc1-pc2}
    \end{figure}

    \item Mon NI est le \texttt{XXX XXX XXX} Voici les adresses assignées:
    
    % Table des addresses
    \begin{center}
        \begin{tabular}{|c|c|}
            \hline
            \textbf{Subnet} & \textbf{Adresse IP} \\
            \hline
            TODO & \texttt{A.B.C.D/X} \\
            \hline
        \end{tabular}
    \end{center}

    \item L'adresse IP choisie pour le PC1 est \texttt{A.B.C.D/X} parce que... Le ping ...
          parce que ...
    \item L'adresse IP choisie pour le PC4 est \texttt{A.B.C.D/X} parce que... Le ping ...
          parce que ...

\end{enumerate}

\section*{Adresses locales IPv6}

\begin{enumerate}
    \item Le ping ... parce que ...
    \item Le ping ... parce que ...
    \item Le ping ... parce que ...
    \item J'en conclus que...
\end{enumerate}

\section*{Tables d'apprentissage du commutateur}

\begin{enumerate}
    \item Il existe des entrées pour... mais pas pour ... et ça a du sens parce que ...
    \item Il existe des entrées pour... mais pas pour ... et ça a du sens parce que ...
    \item Les entrées sont ... parce que ...

    \item L'adresse IP source du ping est \texttt{A.B.C.D} alors que l'adresse destination
          est \texttt{E.F.G.H}. Ces adresses correspondent à ... et ... Ces adresses ... en
          transit parce que ...
    \item L'adresse MAC source du ping est \texttt{XX:XX:XX:XX:XX:XX} alors que l'adresse
          MAC destination est \texttt{YY:YY:YY:YY:YY:YY}. Ces adresses correspondent à ... et ...
          Ces adresses ... en transit parce que ...
    \item J'en conclus que ...
    \item J'en conclus que ...
\end{enumerate}

\section*{ARP}

\begin{enumerate}
    \item Le commutateur ... parce que ...
    \item Voici les entrées dans la table ARP du commutateur S1:
    
    \begin{itemize}
        \item \texttt{A.B.C.D} $\rightarrow$ \texttt{XX:XX:XX:XX:XX:XX}: Cette entrée correspond
              à ... et ça a du sens qu'elle soit là parce que ...
    \end{itemize}

    Ou sinon la table est vide parce que ...
    \item Voici les entrées dans la table ARP du commutateur PC1:
    
    \begin{itemize}
        \item \texttt{A.B.C.D} $\rightarrow$ \texttt{XX:XX:XX:XX:XX:XX}: Cette entrée correspond
              à ... et ça a du sens qu'elle soit là parce que ...
    \end{itemize}

    Ou sinon la table est vide parce que ...
\end{enumerate}

\section*{DHCP}

\begin{enumerate}
    \item Ma capture Wireshark est présentée à la figure~\ref{fig:wireshark-dhcp}.

    \begin{figure} 
        \centering
        \includegraphics[width=0.8\textwidth]{} % TODO
        \caption[]{Capture Wireshark d'un échange DHCP.}
        \label{fig:wireshark-dhcp}
    \end{figure}

    \item Les datagrammes ... parce que ...

    \item Les différents paramètres envoyés sont les suivants:

    \begin{itemize}
        \item Param 1: Ce paramètre est utile parce que ...
    \end{itemize}

    Il pourrait aussi y avoir d'autres paramètres comme ...

    \item L'adresse source était \texttt{A.B.C.D} et l'adresse destination était
          \texttt{E.F.G.H}. Ces adresses correspondent à ... et ... Elles sont
          utilisées parce que ...

\end{enumerate}

\section*{ARP (suite)}

\begin{enumerate}
    \item L'état de la table ARP de S1 est présenté à la figure~\ref{fig:arp-table-s1}.

    \begin{figure} 
        \centering
        \begin{lstlisting}
$ arp -a
? (A.B.C.D) at XX:XX:XX:XX:XX:XX [ether] on eth0
? (E.F.G.H) at YY:YY:YY:YY:YY:YY [ether] on eth0
        \end{lstlisting}
        \caption[]{Table ARP de S1.}
        \label{fig:arp-table-s1}
    \end{figure}

    Voici ce à quoi correspondent les entrées:

    \begin{itemize}
        \item \texttt{A.B.C.D} $\rightarrow$ \texttt{XX:XX:XX:XX:XX:XX}: Cette entrée correspond
              à ... et ça a du sens qu'elle soit là parce que ...
        \item \texttt{E.F.G.H} $\rightarrow$ \texttt{YY:YY:YY:YY:YY:YY}: Cette entrée correspond
              à ... et ça a du sens qu'elle soit là parce que ...
    \end{itemize}

    Ou sinon la table est vide parce que ...

    \item L'était de la table ARP de PC1 est présentée à la figure~\ref{fig:arp-table-pc1}.
    \begin{figure}
        \centering
        \begin{lstlisting}
$ arp -a
? (A.B.C.D) at XX:XX:XX:XX:XX:XX [ether] on eth0
? (E.F.G.H) at YY:YY:YY:YY:YY:YY [ether] on eth0
        \end{lstlisting}
        \caption[]{Table ARP du PC1.}
        \label{fig:arp-table-pc1}
    \end{figure}

    Voici ce à quoi correspondent les entrées:

    \begin{itemize}
        \item \texttt{A.B.C.D} $\rightarrow$ \texttt{XX:XX:XX:XX:XX:XX}: Cette entrée correspond
              à ... et ça a du sens qu'elle soit là parce que ...
        \item \texttt{E.F.G.H} $\rightarrow$ \texttt{YY:YY:YY:YY:YY:YY}: Cette entrée correspond
              à ... et ça a du sens qu'elle soit là parce que ...
    \end{itemize}

    Ou sinon la table est vide parce que ...

\end{enumerate}

\section*{IP et Ethernet}

\subsection*{IP}

\begin{enumerate}
    \item L'adresse MAC source de la trame entre PC1 et S1 est \texttt{XX:XX:XX:XX:XX:XX} et
          l'adresse MAC destination est \texttt{YY:YY:YY:YY:YY:YY}.
    \item L'adresse MAC source de la trame entre S1 et S2 est \texttt{XX:XX:XX:XX:XX:XX} et
          l'adresse MAC destination est \texttt{YY:YY:YY:YY:YY:YY}.
    \item L'adresse MAC source de la trame entre S2 et R1 est \texttt{XX:XX:XX:XX:XX:XX} et
          l'adresse MAC destination est \texttt{YY:YY:YY:YY:YY:YY}.
    \item L'adresse MAC source de la trame entre R1 et PC4 est \texttt{XX:XX:XX:XX:XX:XX} et
          l'adresse MAC destination est \texttt{YY:YY:YY:YY:YY:YY}.
    \item Les adresses ont changé entre aux étapes x et y parce que ...
          Ou elles n'ont jamais changé parce que ...
\end{enumerate}

\subsection*{Ethernet}

\begin{enumerate}
    \item L'adresse IP source du datagramme entre PC1 et S1 est \texttt{A.B.C.D} et
          l'adresse IP destination est \texttt{E.F.G.H}.
    \item L'adresse IP source du datagramme entre S1 et S2 est \texttt{A.B.C.D} et
          l'adresse IP destination est \texttt{E.F.G.H}.
    \item L'adresse IP source du datagramme entre S2 et R1 est \texttt{A.B.C.D} et
          l'adresse IP destination est \texttt{E.F.G.H}.
    \item L'adresse IP source du datagramme entre R1 et PC4 est \texttt{A.B.C.D} et
          l'adresse IP destination est \texttt{E.F.G.H}.
    \item Les adresses ont changé entre aux étapes x et y parce que ...
            Ou elles n'ont jamais changé parce que ...
\end{enumerate}
    
\end{document}